\documentclass[11pt]{scrartcl}

\usepackage[utf8]{inputenc}
\usepackage{amsmath}
\usepackage{biblatex}
\usepackage{caption}
\usepackage{float}
\usepackage{graphicx}
\usepackage{subcaption}

\renewcommand{\bibfont}{\footnotesize}
%\pagenumbering{gobble}
\usepackage{hyperref}
\addbibresource{report.bib}

\title{An Investigation of Phases in Context Tree Weighting}
\subtitle{CS281B: Final Project Report}
\author{Constantin Berzan (joint work with Eric Tzeng)}
\date{May 1, 2014}

\begin{document}
\maketitle

\begin{abstract}
TODO
\end{abstract}

%%%%%%%%%%%%%%%%%%%%%%%%%%%%%%%%%%%%%%%%%%%%%%%%%%%%%%%%%%%%%%%%%%%%%%%%%%%%%%
\section{Introduction}


%%%%%%%%%%%%%%%%%%%%%%%%%%%%%%%%%%%%%%%%%%%%%%%%%%%%%%%%%%%%%%%%%%%%%%%%%%%%%%
\section{The Binary Weighted Context Tree (WCTBinary)}


%%%%%%%%%%%%%%%%%%%%%%%%%%%%%%%%%%%%%%%%%%%%%%%%%%%%%%%%%%%%%%%%%%%%%%%%%%%%%%
\section{The Weighted Context Tree with Phases (WCTPhases)}


%%%%%%%%%%%%%%%%%%%%%%%%%%%%%%%%%%%%%%%%%%%%%%%%%%%%%%%%%%%%%%%%%%%%%%%%%%%%%%
\section{The Weighted Context Tree on Bytes (WCTBytes)}


%%%%%%%%%%%%%%%%%%%%%%%%%%%%%%%%%%%%%%%%%%%%%%%%%%%%%%%%%%%%%%%%%%%%%%%%%%%%%%
\section{Plaintext Transformations}

English text represented in ASCII exhibits two properties that might be
exploited by a compression algorithm: locality and byte alignment. Locality
means that commonly used characters have similar representations (the
lower-case letters range from {\tt 01100001} to {\tt 01111010}, and the
upper-case letters range from {\tt 01000001} to {\tt 01011010}). Byte alignment
means that each character is represented as exactly one byte.

To investigate how much our compression algorithms rely on these properties, we
devised two simple transformations that disrupt these properties. The {\bf
scrambling} transformation passes the text through a random 1-1 mapping from
bytes to bytes, thus disrupting locality. The {\bf misaligning} transformation
prepends a zero bit to every byte, thus making each character occupy nine bits,
and disrupting byte alignment. Figure \ref{fig:transformations} summarizes
these transformations in pseudocode.

\begin{figure}[h!]
    \centering
    \begin{subfigure}[b]{0.45\textwidth}
\begin{verbatim}
def scramble(input):
    mapping = rand_permut(256)
    for char in input:
        output mapping[char] \end{verbatim}
        \caption{Scrambling transformation}
    \end{subfigure}
    ~
    \begin{subfigure}[b]{0.45\textwidth}
\begin{verbatim}
def misalign(input):
    for byte in input:
        output 0
        output all bits of byte \end{verbatim}
        \caption{Misaligning transformation}
    \end{subfigure}
    \caption{Pseudocode for the plaintext transformations.}
    \label{fig:transformations}
\end{figure}



%%%%%%%%%%%%%%%%%%%%%%%%%%%%%%%%%%%%%%%%%%%%%%%%%%%%%%%%%%%%%%%%%%%%%%%%%%%%%%
\section{Results and Discussion}

We ran experiments on an English text document of 1270 characters. (It is the
first paragraph from Harlan Ellison's short story, {\em ``Repent, Harlequin!''
Said the Ticktockman}, which is in turn a quotation from Henry David Thoreau's
essay, {\em Civil Disobedience}.)

We validated our implementation in two ways: First, we verified that encoding
and then decoding the text produced the original text back. Second, we verified
that our implementation of WCTPhases obtained the same compression ratio as the
official CTW implementation run with the KT estimator. Thus, we are confident
that our implementation of WCTPhases is equivalent to the official CTW
implementation.

Our implementation was not nearly as fast as the official CTW implementation.
Encoding and decoding our example text took 5.8 seconds with WCTBinary, and
21.5 seconds with WCTPhases. The official CTW implementation took 0.05 seconds
for the same task (430 times faster than WCTPhases). These runtimes are
reported for a modern laptop with a 2.9 GHz CPU. The official CTW
implementation is written in C and includes multiple optimizations for speed
(single-path pruning, pre-computed logarithm tables, and a more advanced
arithmetic encoder). In our Python implementation, we focused on simplicity and
easy experimentation, rather than speed.

For our main experiment, we compared the compression ratio obtained by
WCTBinary, WCTPhases, and WCTBytes on the original text, a scrambled version of
the original text, and a misaligned version of the original text. All trees
were configured with a depth of 6 bytes, or 48 bits, which is the default value
used in the official CTW implementation. Table \ref{tab:results} summarizes our
results.

\begin{table}[h!]
    \centering
    \begin{tabular}{l|lll}
        & plain text & scrambled text & misaligned text \\
        \hline
        WCTBinary & 4.69 bits/byte & 5.41 bits/byte & 4.64 bits/byte \\
        WCTPhases & 3.90 bits/byte & 4.12 bits/byte & 6.87 bits/byte \\
        WCTBytes  & 4.61 bits/byte & 4.61 bits/byte & 7.45 bits/byte \\
    \end{tabular}
    \caption{Compression ratio for the various algorithms and plaintext
    transformations.}
    \label{tab:results}
\end{table}

WCTBinary did equally well for the plain text and the misaligned text, but it
did poorly on the scrambled text. This tells us that WCTBinary takes advantage
of the locality of ASCII, but does not take advantage of byte-aligned data.

WCTPhases did comparably well for the plain text and the scrambled text, but it
did much worse on the misaligned text. This tells us that WCTPhases does not
rely on locality as much, but it does rely heavily on byte-aligned data.

WCTBytes was unaffected by scrambling at all, but it did much worse on
misaligned text. This was to be expected, since out of all three methods,
WCTBytes is the most reliant on byte-aligned data.

We verified that the results we just presented hold for other English text
documents of similar length. For longer documents, the compression ratios are
better, but the effect of scrambling and misalignment remain qualitatively the
same.

% TODO: other things to maybe talk about:
% - space symbol appearing in WCTBinary in different nodes
% - WCTPhases predicts the first bit of ASCII chars for free (always 0)
% - experiments with changing the weighting


%%%%%%%%%%%%%%%%%%%%%%%%%%%%%%%%%%%%%%%%%%%%%%%%%%%%%%%%%%%%%%%%%%%%%%%%%%%%%%
\section{Conclusion and Future Work}


%%%%%%%%%%%%%%%%%%%%%%%%%%%%%%%%%%%%%%%%%%%%%%%%%%%%%%%%%%%%%%%%%%%%%%%%%%%%%%
\printbibliography


\end{document}
